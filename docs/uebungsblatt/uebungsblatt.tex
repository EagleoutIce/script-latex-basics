\documentclass{article} % Wir schreiben ein Übungsblatt

\usepackage[T1]{fontenc} % Kodierung des Texts
\usepackage[utf8]{inputenc} % Damit die Dateikodierung egal ist

\usepackage[ngerman]{babel} % ngerman = new german = neue deutsche Rechtschreibung. Wortrennungen und so

\usepackage{lmodern} % Schriftdarstellung in PDF verbessert
\usepackage{microtype} % Ausgeglichener Grauwert

\usepackage{amsmath} % Mathe gekrarmse
\usepackage{enumitem} % Modifikation der enumerate-Umgebung

% schönere Seitenränder:
\usepackage[margin=2cm,a4paper]{geometry}


\newcounter{taskctr}

% #1 Name
% #2 Punkte
\newenvironment{aufgabe}[2][]{\refstepcounter{taskctr}\bgroup%
\leavevmode\newline\noindent\textbf{Aufgabe~\thetaskctr}\ifx\\#2\\\else:~#2\fi%
\ifx\\#1\\\else\hfill\textit{(#1)}\fi\smallskip\par\noindent\ignorespaces%
}{\egroup\bigskip}

\newlist{aufgaben}{enumerate}{1}
\setlist[aufgaben]{label=\textit{\alph*)},topsep=0pt}






%%%%%%%%%%%%%%%%%%%%%%%%%%%%%%%%%%%%%%%%%%%%%%%%%%%%%%%%%%%%%%%%%%%%%%%%%%%%%%%%%%%%%%%%%%%%%%%%%%%5





% Damit beginnt das Dokument:
\begin{document}

\title{Übungsblatt 42}
\author{Ente eins\\\texttt{ente.eins@uni-ulm.de} \and
        Ente zwei\\\texttt{ente.zwei@uni-ulm.de}}
\date{01.02.4242}

\maketitle

\begin{aufgabe}{Tanze mit den Sternen}
    \begin{aufgaben}
        \item Hier haben wir mit den Sternen Händchen gehalten
        \item Gibt es noch Brot? Hallo?
    \end{aufgaben}
\end{aufgabe}

\begin{aufgabe}[4.5 Punkte]{Eine Ente kommt selten allein}
    Diese Aufgabe war sehr schwierig, zuerst haben wir gerechnet: \(42 \cdot 3^{e + \sin(x)}\), dann geweint und schließlich die Lösung gefunden: 42 Quacks.
\end{aufgabe}

\begin{aufgabe}{}
    Ahnungsloses Quack.
\end{aufgabe}

\end{document}
% Und hier isses dann auch schon vorbei.